\documentclass[
a4paper,
oneside,
halfparskip*,
normalheadings,
]
{scrbook}

\usepackage[english]{babel}
\usepackage[ansinew]{inputenc}
\usepackage{graphicx}

\begin{document}

\title{HE853 Control Documentation}
\author{Thomas Ascher}
\maketitle
\tableofcontents
\frontmatter
\mainmatter

\chapter{Introduction}

The HE853 Control project is dedicated to create a Windows SDK for the HE853 USB
dongle of the Home Easy home automation products since the vendor does not provide
one.

The SDK consists of GPL licensed tools and a LGPL licensed library that can be
used from other applications.

Out target audience are power users and software developers who whish to extend the
usage of the HE853 USB dongle beyond the limits of the vendor software.

Most information this project is based on was gathered through reverse engineering.
So not all details of the used communiation protocol are known and doumented.

\chapter{Tools}

\section{Command Line Utility}

\section{GUI Application}

\section{Service}

The service HE853.Service is for advanced users. It is meant for scenare where multiple
applications have to access the HE853 dongle concurrently. Both the utility and the
GUI application support the \textit{/service} switch that makes them use the service
instead of using the device directly.

\subsection{Configuration}

\subsection{Manual Installation}
To install the service you have to use Installer Tool (Installutil.exe). The service is
configured to not start automatically per default. 



\chapter{Development}

\section{Using the Library}

\subsection{CSharp Sample Code}
\begin{verbatim}
HE853.IDevice device = new HE853.Device();
if (device.Open())
{
    device.On(1001);
    device.Off(1001);
    device.Close();
}
\end{verbatim}



\section{Depolyment}
One can simply install the HE853 Control MSI setup on a 

\subsection{Manual COM Registration}

\chapter{HE853 Control Development}



\end{document}