\documentclass[
a4paper,
oneside,
halfparskip*,
normalheadings,
]
{scrbook}

\usepackage[english]{babel}
\usepackage[ansinew]{inputenc}
\usepackage{graphicx}
\usepackage{url}

\begin{document}

\title{HE853 Control Documentation}
\author{Thomas Ascher}
\maketitle
\tableofcontents
\mainmatter

\chapter{Introduction}

The HE853 Control project is dedicated to create a Windows SDK for the HE853 USB
dongle of the Home Easy home automation products since the vendor does not provide
one.

The SDK consists of GPL licensed tools and a LGPL licensed library that can be
used from other applications.

Out target audience are power users and software developers who whish to extend the
usage of the HE853 USB dongle beyond the limits of the vendor software.

Most information this project is based on was gathered through reverse engineering.
So not all details of the used communiation protocol are known and documented.

For more information about the Home Easy home automation products please refer to
the vendor page: \url{http://www.elro.eu/en/products/category/home_automation/home_easy/zenders2/pc_afstandsbediening_usb_dongle}

For more information and the latest packages please refer to the project page:
\url{http://he853control.sourceforge.net/}

\section{Prequesites}
For installation, usage and development Microsoft .NET Framework 2.0 has to be installed on
your system: \url{http://go.microsoft.com/fwlink/?LinkId=131000}




\chapter{Tools}

All tools are located in the installation directory and support the following commands:

\begin{itemize}
	\item On: receiver will perform switch on
	\item Off: receiver will perform switch off
	\item Dim 10--80: reciver will perform dim to specific level if supported
\end{itemize}


\section{Command Line Utility -- HE853.Util.exe}

\includegraphics[width=300px]{cmd.png}

Usage: HE853.Util \textless command\textgreater \textless device\textunderscore code\textgreater [/service]

\textless command\textgreater := ON | OFF | 10..80
\textless device\textunderscore code\textgreater := 1..6000

Example: HE853.Util ON 1001

\section{GUI Application -- HE853.App.exe}

\includegraphics[width=100px]{gui.png}

Enter a specifi device code and click the buttons to send a specific command.

\section{Coupling Dongle with Receivers}

Before a receiver react to commands of the USB dongle a specific device code has to be
programmed to the receiver first. A device code is a number between 1 and 6000 and can
be used by multiple receivers.

To program a device code use the command line utility or the GUI application. Press the
learn button on a receiver for about 1 second. Then send an ON command with the desired
device code from the command line utility or the GUI application. The receiver will
react to commands with this device code.

To clear the programmed device code press the learn button on a receiver util the LED
on the receiver starts to flash.

\section{Service -- HE853.Service.exe}

The service HE853.Service is for advanced users. It is meant for scenarios where multiple
applications have to access the HE853 dongle concurrently. Both the utility and the
GUI application support the \textit{/service} switch that makes them use the service
instead of using the device directly.

\subsection{Configuration}

The service is configured to start manually per default. This behavior has to be changed
in the service control panel. Configure the HE853 RPC Service to start automatically and
or start it. 

\subsection{Manual Installation}

If the service was not installed with the MSI installer is has to be registered with
the Installer Tool (Installutil.exe).





\chapter{Development}

For development at least Microsoft .NET Framework 2.0 and or Visual Studio 2005 is required.

\section{Using the Library -- HE853.dll}

The library can be used by any .NET language by referencing the HE853.dll assembly from the
installation directory or via COM. Please note that the assembly is compiled as Any CPU
and works with x86 and x64 platform configurations.

Visual Studio 2010 sample projects for the languages C++, C++/CLI, C\# and Visual Basic
are located in the installation folder in \textit{\textbackslash Samples}.

Please note that the library is licensed under the term of the LGPL. This means it is
possible to use the library in commercial applications but code changes to the library
have to be made available on distribution.

\subsection{C\# Sample Code}

\begin{verbatim}
HE853.IDevice device = new HE853.Device();
if (device.Open())
{
    device.On(1001);
    device.Off(1001);
    device.Close();
}
\end{verbatim}

\section{Using the RPC Library -- HE853.RPC.dll}

The RPC library ist not designed for general usage yet and is not available via COM.
It uses .NET Remoting IPC for communication. The library does only work in combination
with the service application.

Use the folloging code in combination with the RPC library:

\begin{verbatim}
HE853.RPC.RegisterClient();
HE853.IDevice device = new HE853.Device();
if (device.Open())
{
    device.On(1001);
    device.Off(1001);
    device.Close();
}
\end{verbatim}

\section{Depolyment}
The easiest method of depolyment is by installing the HE853 Control MSI setup on the target
system. The installation performs all required steps:

\begin{itemize}
  \item Installation of HE853.dll to the global assembly cache
	\item Registration of HE853.dll for COM usage
	\item Registration of HE853.Service.exe
\end{itemize}

If you deploy .NET applications you can simply install or copy the HE853.dll with your application
as private assembly.




\chapter{Support HE853 Control}

You can support this project by testing, writing code or documentation and suggesting features
and reporting bugs.
For writing documentation a good knowledge of LaTeX is required. For writing code one should
have knowledge of:

\begin{itemize}
	\item Visual Studio 2010 (required for development)
	\item C\#
	\item PInvoke and Windows API
	\item HID Devices
	\item Unit Tests
\end{itemize}

\section{TODOs}

\begin{itemize}
  \item Correct spelling errors in documentation
	\item Improve user documentation
	\item Improve source documentation (at least for public interfaces)
	\item Better input validation in program
	\item Rewrite code to use exceptions instead of return values
	\item Refactor code so it is more suiteable for test driven development
	\item Write more unit tests
	\item Add installation path to PATH variable during installation
	\item Make Command classes internal again, exposed now for unit tests
	\item Test on more 32 and 64 bit platforms: XP, Vista, Windows 7, Windows 8
	\item Implement GUI as custom control
	\item Improve GUI application (help integration, layout, etc.)
	\item Maybe translate the package into other languages
\end{itemize}

\end{document}